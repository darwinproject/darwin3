\documentclass[11pt,letterpaper,english]{article}
\usepackage[T1]{fontenc}
\usepackage[latin1]{inputenc}
\setlength\parskip{\medskipamount}
\setlength\parindent{0pt}
\usepackage{amsmath}
%\usepackage{graphicx}
\usepackage{amssymb}
\usepackage{babel}
%use epsfig package for figs
\usepackage{epsfig}
\makeatother


%MICK - STUFF TO USE HELVETICA FONTS IN TEX
%%\usepackage[scaled=0.92]{helvet}
\usepackage{helvet}
\usepackage[sf]{titlesec}
\renewcommand\familydefault{\sfdefault}

%use lgrind to include code listing
%\usepackage{lgrind}

%other formatting stuff
\oddsidemargin 0pt
\flushbottom
\parskip 10pt
\parindent 0pt
\textwidth 465pt
\topmargin 10pt
\textheight 610pt
\renewcommand{\baselinestretch}{1.0}


\begin{document}

% SOME MACROS
\newcommand{\etal}{{\em et al.}}
\newcommand{\ux}{{\underline{x}}}
\newcommand{\tdt}{{t}}


{\bf {\large A1. Ecosystem Model Parametrization}}

The ecosystem model equations are similar that used in Follows et al. (2007).
Most significant change is that the grazing term is now includes variable
palatibility of phytoplankton and sloppy feeding as treated in Dutkiewicz et al
(2005), and the nitrogen limitation term has been slightly modified.  For
fuller discussions we refer the reader to the Online Supplemental material of
Follows et al. (2007).

Several nutrients $N_i$ nourish many phytoplankton types $P_j$ which are grazed
by several zooplankton types $Z_k$. Mortality of and excretion from plankton,
and sloppy feeding by zooplankton contribute to a dissolved organic matter
$DOM_i$ pool and a sinking particulate organic matter pool $POM_i$.  Subscript
$i$ refers to a nutrient/element, $j$ for a specific phytoplankton type, and
$k$ for a zooplankton type.

\begin{eqnarray}
\frac{\partial N_i}{\partial t} & = & 
-\nabla \cdot (\textbf{u} N_i) +\nabla \cdot (\kappa\nabla N_i)-
\sum_j [\mu_j P_j M_{ij}]+S_{N_i}
\nonumber  \\
\frac{\partial P_j}{\partial t} & = & 
-\nabla \cdot (\textbf{u} P_j) + \nabla \cdot (\kappa\nabla P_j)+
\mu_j P_j - m_j^P P_j-\sum_k [g_{jk} Z_{k,i=1}]
-\frac{\partial(w_j^P P_j)}{\partial z}
\nonumber \\
\frac{\partial Z_{ki}}{\partial t} & = & 
- \nabla \cdot (\textbf{u} Z_{ki}) + \nabla \cdot (\kappa\nabla Z_{ki})
+Z_{ki}\sum_j [\zeta_{jk} g_{jk} M_{ij}] -m_k^Z Z_{ki}
\nonumber \\
\frac{\partial POM_i}{\partial t} & = & 
-\nabla \cdot (\textbf{u} POM_i) + \nabla \cdot (\kappa\nabla POM_i)-
r_{PO_i}POM_i-\frac{\partial(w_{POi} POM_i)}{\partial z}+S_{POM_i}
\nonumber \\
\frac{\partial DOM_i}{\partial t} & = & 
-\nabla \cdot (\textbf{u} DOM_i) + \nabla \cdot (\kappa\nabla DOM_i)- 
r_{DO_i}DOM_i + S_{DOM_i} 
\nonumber
\end{eqnarray}

where:\\
\mbox{} \hspace{.5cm} $\textbf{u}=(u,v,w),$ velocity in physical model, \\
\mbox{} \hspace{.5cm} $\kappa=$Mixing coefficients used in physical model,\\
\mbox{} \hspace{.5cm} $\mu_j=$Growth rate of phytoplankton $j$ (see below),\\
\mbox{} \hspace{.5cm} $M_{ij}=$Matrix of Redfield ratio of element $i$ to P
for phytoplankton $j$\\
\mbox{} \hspace{.5cm} $\zeta_{jk}=$ Grazing efficiency of zooplankton $k$ on phytoplankton $j$ (represents sloppy feeding), \\
\mbox{} \hspace{.5cm} $g_{jk}=$Grazing of zooplankton $k$ on phytoplankton $j$ (see below),\\
\mbox{} \hspace{.5cm} $m_j^P=$Mortality/Excretion rate for phytoplankton $j$,\\
\mbox{} \hspace{.5cm} $m_k^Z=$Mortality/Excretion rate for zooplankton $k$,\\
\mbox{} \hspace{.5cm} $w_j^P=$Sinking rate for phytoplankton $j$,\\
\mbox{} \hspace{.5cm} $w_{POi}=$Sinking rate for POM $i$,\\
\mbox{} \hspace{.5cm} $r_{DOM_i}=$Remineralization rate of DOM for element
$i$,\\
\mbox{} \hspace{.5cm} $r_{POM_i}=$Remineralization rate of POM for element
$i$,\\
\mbox{} \hspace{.5cm} $S_{N_i}=$Additional source or sink for nutrient $i$
(see below),\\
\mbox{} \hspace{.5cm} $S_{DOM_i}=$ Source of DOM $i$,
for element $i$ (see below),\\
\mbox{} \hspace{.5cm} $S_{POM_i}=$ Source of POM $i$,
for element $i$ (see below),\\



\vspace{.2cm}

{\it {\bf A1.1. Phytoplankton growth:}}\\
\[
\mu_j = \mu_{max_{j}} \gamma_j^T \gamma_j^I \gamma_j^N
\]
where\\
\mbox{} \hspace{.5cm} $\mu_{max_{j}}=$ maximum growth rate of phytoplankton $j$,\\
\mbox{} \hspace{.5cm} $\gamma_j^T=$Modification of growth rate by
temperature for phytoplankton $j$,\\
\mbox{} \hspace{.5cm} $\gamma_j^I=$Modification of growth rate by light for
phytoplankton $j$,\\
\mbox{} \hspace{.5cm} $\gamma_j^N=$Modification of growth rate by nutrients
for phytoplankton $j$.\\

Temperature modification (Fig. \ref{fig-growexp1}a):\\
\[
\gamma_j^T= \frac{1}{\tau_1} (A^T e^{-B(T-T_o)^c} - \tau_2 )
\]
where coefficients $\tau_1$ and $\tau_2$ normalize the maximum
value, and $A,B,T_o$ and $C$ regulate the form of the temperature
modification function. $T$ is the local model ocean temperature.

Light modification (Fig. \ref{fig-growexp1}b):\\
\[
\gamma_j^I= \frac{1}{F_o} (1-e^{k_{par} I} ) e^{-k_{inhib} I}
\]
where $F_{o}$ is a factor controlling the maximum value, $k_{par}$ is the
PAR saturation coefficient and $k_{inhib}$ is the PAR inhibition factor.
$I$ is the local PAR, that has been attenuated through the water column
(including the effects of self-shading).

Nutrient limitation is determined by the most limiting nutrient:
\[
\gamma_j^N = \min(N_i^{lim})
\]
where typically
$N_i^{lim}=\frac{N_i}{N_i+\kappa_{N_{ij}}}$
(Fig. \ref{fig-growexp1}c) and $\kappa_{N_{ij}}$ is the half saturation constant of nutrient $i$ for phytoplankton $j$.

When we include the nitrogen as a potential limiting nutrient (EXP2) we 
modify $N_i^{lim}$ to take into account the uptake inhibition caused by ammonium:
\[
N_N^{lim} = \frac{NO_3 + NO_2}{NO_3+NO_2+\kappa_{IN}} e^{-\psi NH_4}
+\frac{NH_4}{NH_4 + \kappa_{NH4}}
\]
where $\psi$ reflects the inhibition and $\kappa_{IN}$ and $ \kappa_{NH4}$
are the half saturation constant of $IN=NO_3+NO_2$ and $NH_4$ respectively.

\vspace{.2cm}

{\it {\bf A1.2. Zooplankton grazing:}}\\
\[
 g_{jk} =g_{max_{jk}} \frac{\eta_{jk} P_j}{A_k} \frac{A_k}{A_k+\kappa^P_k}
\]
where\\
\mbox{} \hspace{.5cm} $g_{max_{jk}}=$ Maximum grazing rate of zooplankton $k$ on
phytoplankton $j$,\\
\mbox{} \hspace{.5cm} $\eta_{jk}=$ Palatibility of plankton $j$ to zooplankton $k$,\\
\mbox{} \hspace{.5cm} $A_k=$ Palatibility (for zooplankton $k$) weighted total phytoplankton concentration,\\
\mbox{} \hspace{1.1cm} $=\sum_j [\eta_{jk} P_j$] \\
\mbox{} \hspace{.5cm} $\kappa^P_k=$Half-saturation constant for grazing of zooplankton $k$,\\


\vspace{.2cm}

{\it {\bf A1.3. Inorganic nutrient Source/Sink terms:}}\\
$S_{N_i}$ depends on the specific nutrient, and includes the remineralization
of organic matter, external sources and other non-biological transformations:
\begin{eqnarray}
S_{PO4} & = & r_{DOP} DOP + r_{POP} POP \nonumber \\
S_{Si}  & = & r_{POSi} POSi \nonumber \\
S_{FeT} & = &  r_{DOFe} DOFe + r_{POFe} POFe -c_{scav} Fe' + \alpha F_{atmos} \nonumber \\
S_{NO3} & = &  \zeta_{NO3} NO_2 \nonumber \\
S_{NO2} & = &  \zeta_{NO2} NH4 - \zeta_{NO3} NO_2 \nonumber \\
S_{NH4} & = &  r_{DON} DON + r_{PON} PON - \zeta_{NO2} NH_4 \nonumber
\end{eqnarray}

where:\\
\mbox{} \hspace{.5cm} $r_{DOM_i}=$Remineralization rate of DOM for element
$i$, here P, Fe, N,\\
\mbox{} \hspace{.5cm} $r_{POM_i}=$Remineralization rate of POM for element
$i$, here P, Si, Fe, N,\\
\mbox{} \hspace{.5cm} $c_{scav}=$scavenging rate for free iron,\\
\mbox{} \hspace{.5cm} $Fe'=$free iron, modelled as in Parekh et al (2004), \\
\mbox{} \hspace{.5cm} $alpha=$solubility of iron dust in ocean water, \\
\mbox{} \hspace{.5cm} $F_{atmos}=$atmospheric deposition of iron dust on surface of model ocean,\\
\mbox{} \hspace{.5cm} $\zeta_{NO3}=\zeta_{NO3}^0(1-I/I_0)_+=$oxidation rate of NO$_2$ to NO$_3$,\\
\mbox{} \hspace{.5cm} $\zeta_{NO2}=\zeta_{NO2}^0(1-I/I_0)_+=$oxidation rate of NH$_4$ to NO$_2$ (is photoinhibited),\\
\mbox{} \hspace{.5cm} $I_0=$critical light level below which oxidation occurs,\\

The remineralization timescale $r_{DOi}$ and $r_{POi}$ parameterizes the break
down of organic matter to an inorganic form through the microbial loop.


{\it {\bf A1.3.1 Fe chemistry:}}\\
\begin{eqnarray}
Fe' & = & FeT - FeL \nonumber \\
FeL & = & L_{tot} -
  \frac{ L_{stab} (L_{tot} - FeT) - 1
        +\sqrt{(1 - L_{stab} (L_{tot} - FeT))^2 + 4 L_{stab} L_{tot}}}
     {2 L_{stab}} \nonumber
\end{eqnarray}
($Fe'$ may be constrained to be less than $Fe'_{max}$ while preserving $FeT$).

 
{\it {\bf A1.4 DOM and POM Source terms:}}\\
$S_{DOM_i}$ and $S_{POM_i}$ are the sources of dissolved and particulate
organic detritus arising from mortality, excretion and sloppy feeding of the
plantkon. We simply define that a fixed fraction $\lambda_m$ of the the
mortality/excretion term and the non-consumed grazed phytoplankton
($\lambda_g$) go into the dissolved pool and the remainder into the particulate
pool. 
\begin{eqnarray}
S_{DOM_i} & = & \sum_{j} [\lambda_{mp_{ij}} m^p_j P_j M_{ij}] 
             + \sum_{k} [\lambda_{mz_{ik}} m^z_k Z_{ik}]
             + \sum_{k} \sum_{j} [\lambda_{g_{ijk}} (1-\zeta_{jk})
                                        g_{ij} M_{ij} Z_k ]
\nonumber \\
S_{POM_i} & = & \sum_{j} [(1-\lambda_{m_{ij}}) m^p_j P_j M_{ij}]
             + \sum_{k} [(1-\lambda_{mz_{ik}}) m^z_k Z_{ik}]
             + \sum_{k} \sum_{j} [(1-\lambda_{g_{ijk}}) (1-\zeta_{jk})  
                                       g_{ij} M_{ij} Z_k ]
\nonumber
\end{eqnarray}


\newcommand{\pcm}[1]{P^C_{m#1}}
\newcommand{\pcmax}[1]{P^C_{\textrm{MAX}#1}}
\newcommand{\pcarbon}{P^C}
\newcommand{\chltoc}{\theta}
\newcommand{\chltocmax}{\theta^{\textrm{max}}}
\newcommand{\chltocmin}{\theta^{\textrm{min}}}
\newcommand{\alphachl}{\alpha^{\textrm{Chl}}}
\newcommand{\mQyield}{\mathit{mQ}^{\textrm{yield}}}
\newcommand{\RPC}{R^{PC}}
\newcommand{\phychl}{\mathit{Chl}}
\newcommand{\aphychlave}{A^{\mathrm{phy}}_{\mathrm{Chl,ave}}}

{\it {\bf A1.4 Geider light limitation model:}}\\
The phytoplankton growth rate is given by the carbon-specific photosynthesis rate
(rate of carbon synthesized per carbon present),
\[
  \mu_j = \pcarbon_j
\]
The carbon-specific photosynthesis rate
\[
  \pcarbon_j = \pcm{,j} \begin{cases}
     1 - e^{-\alphachl_j I \chltoc_j/\pcm{,j}} & \text{if }I>0.1 \\
     0                                         & \text{otherwise}
   \end{cases}
\]
depends on the carbon-specific, light-saturated photosynthesis rate
\[
  \pcm{,j}=\pcmax{j} \gamma^N_j \gamma^T_j
\]
and the Chl $a$ to carbon ratio
\[
  \chltoc_j = \left[ \frac{\chltocmax_j}
                   {1 + \chltocmax_j \alphachl_j I / (2 \pcm{,j})}
		   \right]^{\chltocmax_j}_{\chltocmin_j}
\]

The chlorophyll concentration is
\[
  \phychl_j=P_j \RPC_j \chltoc_j
\]

The light limitation factor can be diagnosed
\[
  \gamma^I_j=\pcarbon_j/\pcm{,j}
\]

\[
  \alphachl_j = \mQyield_j \aphychlave
\]

Parameters:\\
\begin{tabular}{@{\qquad}r@{}l}
 $\pcmax{j}    ={}$& Maximum C-spec.\ photosynthesis rate at reference temperature of phytoplankton $j$\\
 $\chltocmax_j ={}$& Maximum Chl a to C ratio if phytoplankton $j$\\
 $\RPC_j       ={}$& Carbon to phosphorus (!) ratio of phytoplankton $j$\\
 $\alphachl_j  ={}$& Chl a-specific initial slope of the photosynthesis-light curve\\
 $\mQyield_j   ={}$& slope of the photosynthesis-light curve per absorption\\
 $\aphychlave  ={}$& absorption ($m^{-1}$) per mg Chl a
\end{tabular}



\newcommand{\Ptot}{P_{\mathrm{tot}}}

{\it {\bf A2 Diagnostics:}}\\
Total phytoplankton biomass:
\[
  \Ptot = \sum_j P_j
\]

\begin{tabular}{llll}
 name & definition && units \\
\hline
 \texttt{PhyTot  } & $\Ptot$                                                        && $\mu\mathrm{M\,P}$ \\
 \texttt{PhyGrp1 } & Total biomass of small phytoplankton with $\texttt{nsrc}=1$    && $\mu\mathrm{M\,P}$ \\
 \texttt{PhyGrp2 } & Total biomass of small phytoplankton with $\texttt{nsrc}=2$    && $\mu\mathrm{M\,P}$ \\
 \texttt{PhyGrp3 } & Total biomass of small phytoplankton with $\texttt{nsrc}=3$    && $\mu\mathrm{M\,P}$ \\
 \texttt{PhyGrp4 } & Total biomass of large non-diatoms                             && $\mu\mathrm{M\,P}$ \\
 \texttt{PhyGrp5 } & Total biomass of diatoms                                       && $\mu\mathrm{M\,P}$ \\
 \texttt{PP      } & Primary production                                             && $\mu\mathrm{M\,P}\, \mathrm{s}^{-1}$ \\
 \texttt{Nfix    } & Nitrogen fixation                                              && $\mu\mathrm{M\,N}\, \mathrm{s}^{-1}$ \\
 \texttt{PAR     } & Photosynthetically active radiation                            && $\mu\mathrm{Ein}\, \mathrm{m}^{-2}\,\mathrm{s}^{-1}$ \\
 \texttt{Rstar01 } & $R^*_{\mathrm{PO4}}$ of Phytoplankton species \#1, \dots       && $\mu\mathrm{M\,P}$ \\
 \texttt{Diver1  } & Number of species with $P_j > 10^{-8}\,\mu\mathrm{M\,P}$       & where $\Ptot>10^{-12}$ \\
 \texttt{Diver2  } & Number of species with $P_j > 0.1\%\,  \Ptot$                  & where $\Ptot>10^{-12}$ \\
 \texttt{Diver3  } & Number of species that constitute 99.9\% of $\Ptot$            & where $\Ptot>10^{-12}$ \\
 \texttt{Diver4  } & Number of species with $P_j > 10^{-5} \cdot \max\limits_j P_j$ & where $\Ptot>10^{-12}$ \\
\end{tabular}


\end{document} 
